\documentclass{amsart}

\title{Math notes}
\author{8dcc}

% ------------------------------------------------------------------------------
% Packages
% ------------------------------------------------------------------------------

% Slightly smaller margins
\usepackage[a4paper,
left=4cm,
right=4cm,
top=3cm,
bottom=3cm,
footskip=1cm]{geometry}

% Link sections and subsections
\usepackage[hidelinks]{hyperref}
\hypersetup{linktoc=all}

% Various math utilities
\usepackage{amsmath}

% Graphs
\usepackage{tikz}
% Use Calc library for coordinate calculations
\usetikzlibrary{calc}
% Change default arrow style
\tikzset{>=stealth}

% Different monospace font for code blocks (listings)
\usepackage{inconsolata}

% Remove author and extra info from the headers
\pagestyle{plain}

% New environment adding spacing for tikz pictures
\newenvironment{tikzpicturecenter}
{\vspace{1em}\begin{center}\begin{tikzpicture}}
    {\end{tikzpicture}\end{center}}

% ------------------------------------------------------------------------------
% Document start
% ------------------------------------------------------------------------------

\begin{document}
\maketitle

\tableofcontents
\newpage

% ------------------------------------------------------------------------------
\section{Geometry}
% ------------------------------------------------------------------------------

% ------------------------------------------------------------------------------
\subsection{Magnitude of a vector}
% ------------------------------------------------------------------------------

The magnitude of a vector is the length of the vector, and it's denoted as
$\|v\|$. The formula for calculating the magnitude of a two-dimensional vector
is the following.

\begin{displaymath}
  \|v\| = \sqrt{v_x^2 + v_y^2}
\end{displaymath}

% NOTE: Environment defined above
\begin{tikzpicturecenter}
  \pgfmathsetmacro{\Vx}{3}
  \pgfmathsetmacro{\Vy}{2}

  % Coordinates of points
  \coordinate (o) at (0,0);
  \coordinate (v) at (\Vx,\Vy);

  % Grid
  \draw[thin, gray, dotted] (-1,-1) grid (4,3);

  % Axis
  \draw[->, gray] (o) -- ($(\Vx+0.5,0)$) node[below]{$x$};
  \draw[->, gray] (o) -- ($(0,\Vy+0.5)$) node[left]{$y$};

  % Lines
  \draw[thick] (o) -- (v) node[pos=0.6, above left]{$\|v\|$};
  \draw[thick, red] (o) -- (\Vx,0) node[pos=0.5, below]{$v_x$};
  \draw[thick, blue] (v) -- (\Vx,0) node[pos=0.5, right]{$v_y$};

  % Points in centers
  \filldraw[gray] (o) circle (1pt) node[above left]{$(0,0)$};
  \filldraw (v) circle (1pt) node[above right]{$v$};
\end{tikzpicturecenter}

% ------------------------------------------------------------------------------
\subsection{Distance between two points}
% ------------------------------------------------------------------------------

The distance between two points is the hypotenuse of a right triangle whose two
cathetus are the difference between the $x$ and $y$ coordinates of the two
points.

\begin{tikzpicturecenter}
  \pgfmathsetmacro{\Ax}{1}
  \pgfmathsetmacro{\Ay}{3}
  \pgfmathsetmacro{\Bx}{4}
  \pgfmathsetmacro{\By}{1}

  % Coordinates of points
  \coordinate (a) at (\Ax,\Ay);
  \coordinate (b) at (\Bx,\By);

  % Grid
  \draw[thin, gray, dotted] (0,0) grid (5,4);

  % Lines
  \draw[thick] (a) -- (b) node[pos=0.5, above right]{$d$};
  \draw[thick, blue] (\Ax,\By) -- (a) node[pos=0.5, left]{$\Delta_y$};
  \draw[thick, red] (\Ax,\By) -- (b) node[pos=0.5, below]{$\Delta_x$};

  % Points in centers
  \filldraw (a) circle (1pt) node[above right]{A};
  \filldraw (b) circle (1pt) node[above right]{B};
\end{tikzpicturecenter}

\begin{displaymath}
  d = \sqrt{(B_x - A_x)^2 + (B_y - A_y)^2}
\end{displaymath}

% ------------------------------------------------------------------------------
\subsection{Unit vector}
% ------------------------------------------------------------------------------

A unit vector is a vector of length 1, and it's usually denoted as $u$ or
$\hat{u}$. The normalized or unitary vector $\hat{u}$ of a vector $v$ is a
vector of length 1 with the direction of $v$. The following formula can be used
for normalizing a vector.

\begin{displaymath}
  \hat{u} = \frac{v}{\|v\|}
\end{displaymath}

% ------------------------------------------------------------------------------
\subsection{Gravitational force}
% ------------------------------------------------------------------------------
% TODO: Update with improvements in 8dcc/orbit

The gravitational force of each body is calculated with the following formula.

\begin{displaymath}
  F = G \frac{m_1m_2}{r^2}
\end{displaymath}

Where $G$ is the gravitational constant, $m_1$ and $m_2$ are the mass of
each body, and $r$ is the distance between the objects.

The effect of a force is to accelerate the body. The relationship is the
following.

\begin{displaymath}
  F = m a
\end{displaymath}

Where $F$ is the force, $m$ is the mass and $a$ is the acceleration of
the body. Therefore, the acceleration can be calculated from the force with the
following formula.

\begin{displaymath}
  a = \frac{F}{m}
\end{displaymath}

The force has a direction. It acts towards the direction of the line joining
the centres of the two bodies. We can get the X and Y coordinates of the
acceleration with some trigonometry.

\begin{align*}
  a_x &= a \cos \theta \\
  a_y &= a \sin \theta \\
\end{align*}

Where $a_x$ and $a_y$ are the X and Y accelerations, $a$ is the
acceleration, and $\theta$ is the angle that the line joining the centers make
with the horizontal.

% ------------------------------------------------------------------------------
% Bibliography
% ------------------------------------------------------------------------------

\newpage
\begin{thebibliography}{9}
\bibitem{vector_introduction}
  Frank D and Nykamp DQ. \textit{An introduction to vectors}. From Math
  Insight. Retrieved 23 May 2024, from
  \url{http://mathinsight.org/vector_introduction}
\bibitem{magnitude_vector}
  Nykamp DQ. \textit{Magnitude of a vector definition}. From Math
  Insight. Retrieved 17 Jun 2024, from
  \url{https://mathinsight.org/definition/magnitude_vector}
\bibitem{unit_vector}
  Wikipedia. \textit{Unit vector}. Retrieved 23 May 2024, from
  \url{https://en.wikipedia.org/wiki/Unit_vector}
\end{thebibliography}

\end{document}
