\documentclass{amsart}

\title{Scratch - Math examples in \LaTeX}
\author{8dcc}

% Various math utilities, like align*
\usepackage{amsmath}

% Graphs
\usepackage{tikz}

% Change the spacing between paragraphs
\setlength{\parskip}{\baselineskip}

% Remove author and extra info from the headers
\pagestyle{plain}

% New environment adding spacing for tikz pictures
\newenvironment{tikzpicturecenter}
{\begin{center}\begin{tikzpicture}}
{\end{tikzpicture}\end{center}}

\begin{document}
\maketitle

Simple scratch file for testing math in \LaTeX.

The distance between two points is the hypotenuse of a right triangle whose two
cathetus are the difference between the $x$ and $y$ coordinates of the two
points.

% NOTE: Environment defined above
\begin{tikzpicturecenter}
  % Coordinates of circles
  \coordinate (a) at (1,4);
  \coordinate (b) at (5,1);

  % Grid
  \draw[thin, gray, dotted] (0,0) grid (6,5);

  % Lines
  \draw[thick] (a) -- (b) node[pos=0.5, above right]{$d$};
  \draw[thick, blue] (1,1) -- (a) node[pos=0.5, left]{$\Delta_y$};
  \draw[thick, red] (1,1) -- (b) node[pos=0.5, below]{$\Delta_x$};

  % Points in centers
  \filldraw (a) circle (1pt) node[above right]{A};
  \filldraw (b) circle (1pt) node[above right]{B};
\end{tikzpicturecenter}

\begin{displaymath}
  d = \sqrt{(B_x - A_x)^2 + (B_y - A_y)^2}
\end{displaymath}

The gravitational force of each body is calculated with the following formula.

\begin{displaymath}
  F = G \frac{m_1m_2}{r^2}
\end{displaymath}

Where $G$ is the gravitational constant, $m_1$ and $m_2$ are the mass of
each body, and $r$ is the distance between the objects.

The effect of a force is to accelerate the body. The relationship is the
following.

\begin{displaymath}
  F = m a
\end{displaymath}

Where $F$ is the force, $m$ is the mass and $a$ is the acceleration of
the body. Therefore, to get the acceleration from the force, we can do the
following.

\begin{displaymath}
  a = \frac{F}{m}
\end{displaymath}

The force has a direction. It acts towards the direction of the line joining
the centres of the two bodies. We can get the X and Y coordinates of the
acceleration with some trigonometry.

\begin{align*}
  a_x &= a \cos \theta \\
  a_y &= a \sin \theta \\
\end{align*}

Where $a_x$ and $a_y$ are the X and Y accelerations, $a$ is the
acceleration, and $\theta$ is the angle that the line joining the centers make
with the horizontal.

\end{document}
