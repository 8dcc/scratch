\documentclass{article}

\title{Math notes}
\author{8dcc}
\date{}

% ------------------------------------------------------------------------------
% Packages
% ------------------------------------------------------------------------------

% Slightly smaller margins
\usepackage[
a4paper,
left=4cm,
right=4cm,
top=3cm,
bottom=3cm,
footskip=1cm
]{geometry}

% Link sections and subsections
\usepackage[hidelinks]{hyperref}
\hypersetup{linktoc=all}

% Various math utilities
\usepackage{amsmath}

% Graphs
\usepackage{tikz}
% Use Calc library for coordinate calculations
\usetikzlibrary{calc}
% Use Decorations for drawing braces
\usetikzlibrary{decorations.pathreplacing}
% Change default arrow style
\tikzset{>=stealth}

% Different monospace font for code blocks (listings)
\usepackage{inconsolata}

% Remove author and extra info from the headers
\pagestyle{plain}

% New environment adding spacing for tikz pictures
\newenvironment{tikzpicturecenter}
{\vspace{1em}\begin{center}\begin{tikzpicture}}
    {\end{tikzpicture}\end{center}}

% Darker green color for graphs
\definecolor{my-green}{rgb}{0.0, 0.7, 0.0}

% ------------------------------------------------------------------------------
% Document start
% ------------------------------------------------------------------------------

\begin{document}
\maketitle
\tableofcontents
\clearpage

% ------------------------------------------------------------------------------
\section{Geometry}
% ------------------------------------------------------------------------------

% ------------------------------------------------------------------------------
\subsection{Magnitude of a vector}
% ------------------------------------------------------------------------------

The magnitude of a vector is the length of the vector, and it's denoted as
$\|v\|$. The formula for calculating the magnitude of a two-dimensional vector
is the following.
\begin{equation*}
  \|v\| = \sqrt{v_x^2 + v_y^2}
\end{equation*}

% NOTE: Environment defined above
\begin{tikzpicturecenter}
  \pgfmathsetmacro{\Vx}{3}
  \pgfmathsetmacro{\Vy}{2}

  % Coordinates of points
  \coordinate (o) at (0,0);
  \coordinate (v) at (\Vx,\Vy);

  % Grid
  \draw[thin, gray, dotted] (-1,-1) grid (4,3);

  % Axis
  \draw[->, gray] (o) -- ($(\Vx+0.5,0)$) node[below]{$x$};
  \draw[->, gray] (o) -- ($(0,\Vy+0.5)$) node[left]{$y$};

  % Lines
  \draw[thick] (o) -- (v) node[pos=0.6, above left]{$\|v\|$};
  \draw[thick, red] (o) -- (\Vx,0) node[pos=0.5, below]{$v_x$};
  \draw[thick, blue] (v) -- (\Vx,0) node[pos=0.5, right]{$v_y$};

  % Points in centers
  \filldraw[gray] (o) circle (1pt) node[above left]{$(0,0)$};
  \filldraw (v) circle (1pt) node[above right]{$v$};
\end{tikzpicturecenter}

% ------------------------------------------------------------------------------
\subsection{Distance between two points}
% ------------------------------------------------------------------------------

The distance between two points is the hypotenuse of a right triangle whose two
cathetus are the difference between the $x$ and $y$ coordinates of the two
points.

\begin{tikzpicturecenter}
  \pgfmathsetmacro{\Ax}{1}
  \pgfmathsetmacro{\Ay}{3}
  \pgfmathsetmacro{\Bx}{4}
  \pgfmathsetmacro{\By}{1}

  % Coordinates of points
  \coordinate (a) at (\Ax,\Ay);
  \coordinate (b) at (\Bx,\By);

  % Grid
  \draw[thin, gray, dotted] (0,0) grid (5,4);

  % Lines
  \draw[thick] (a) -- (b) node[pos=0.5, above right]{$d$};
  \draw[thick, blue] (\Ax,\By) -- (a) node[pos=0.5, left]{$\Delta_y$};
  \draw[thick, red] (\Ax,\By) -- (b) node[pos=0.5, below]{$\Delta_x$};

  % Points in centers
  \filldraw (a) circle (1pt) node[above right]{A};
  \filldraw (b) circle (1pt) node[above right]{B};
\end{tikzpicturecenter}

\begin{equation*}
  d = \sqrt{(B_x - A_x)^2 + (B_y - A_y)^2}
\end{equation*}

% ------------------------------------------------------------------------------
\subsection{Unit vector}
% ------------------------------------------------------------------------------

A unit vector is a vector of length 1, and it's usually denoted as $u$ or
$\hat{u}$. The normalized or unitary vector $\hat{u}$ of a vector $v$ is a
vector of length 1 with the direction of $v$. The following formula can be used
for normalizing a vector.
\begin{equation*}
  \hat{u} = \frac{v}{\|v\|}
\end{equation*}

% ------------------------------------------------------------------------------
\subsection{Sine and cosine}
% ------------------------------------------------------------------------------

Given the following right triangle, containing the acute angle $\alpha$:

\begin{tikzpicturecenter}
  % Distance from A to B
  \pgfmathsetmacro{\HypotLength}{2};

  % Radius and angle of the arc
  \pgfmathsetmacro{\ArcRadius}{0.75};
  \pgfmathsetmacro{\Angle}{35};

  % Position of point B
  \pgfmathsetmacro{\BX}{\HypotLength*cos(\Angle)};
  \pgfmathsetmacro{\BY}{\HypotLength*sin(\Angle)};

  % Position for the theta label, in the middle of the angle
  \pgfmathsetmacro{\ArcLabelX}{\ArcRadius*cos(\Angle/2)};
  \pgfmathsetmacro{\ArcLabelY}{\ArcRadius*sin(\Angle/2)};

  % Coordinates
  \coordinate (a) at (0,0);
  \coordinate (b) at (\BX,\BY);
  \coordinate (c) at (\BX,0);

  % Draw the arc
  \filldraw[violet, fill opacity=0.3] (a) -- (\ArcRadius,0) arc (0:\Angle:\ArcRadius) -- cycle;
  \node at (\ArcLabelX, \ArcLabelY) [yshift=0.05cm, xshift=0.2cm] {$\alpha$};

  \draw[-, thick, red] (a) -- (b) node[above left, pos=0.6]{hypotenuse};
  \draw[-, thick, blue] (a) -- (c) node[below, pos=0.5]{adjacent};
  \draw[-, thick, my-green] (b) -- (c) node[right, pos=0.5]{opposite};

  % Draw a dot in each point
  \filldraw (a) circle (1pt) node[left]{A};
  \filldraw (b) circle (1pt) node[right]{B};
  \filldraw (c) circle (1pt) node[right]{C};
\end{tikzpicturecenter}

The sine and cosine of the angle can be calculated with the following formulas:
\begin{align*}
  \sin \left( \alpha \right) &= \frac{\text{opposite}}{\text{hypotenuse}} \\
  \cos \left( \alpha \right) &= \frac{\text{adjacent}}{\text{hypotenuse}}
\end{align*}

Alternatively, the following definition uses a \textbf{unit circle} to visualize
the sine and cosine more clearly. A unit circle is a circle of radius one
centered at the origin (0, 0) in the cartesian coordinate system.

By tracing a line from the origin to a point in this circle, an angle $\theta$
is formed with the positive \textsl{x} axis. The \textsl{x} and \textsl{y}
coordinates of this point are equal to $\cos \theta$ and $\sin \theta$,
respectively.

\begin{tikzpicturecenter}
  % Radius for the dotted circumference and the angle
  \pgfmathsetmacro{\Radius}{2};
  \pgfmathsetmacro{\ArcRadius}{1};

  % Degrees of angle theta
  \pgfmathsetmacro{\Angle}{40};

  % Point where the black line ends
  \pgfmathsetmacro{\LineEndX}{\Radius*cos(\Angle)};
  \pgfmathsetmacro{\LineEndY}{\Radius*sin(\Angle)};

  % Position for the theta label, in the middle of the angle
  \pgfmathsetmacro{\ArcLabelX}{\ArcRadius*cos(\Angle/2)};
  \pgfmathsetmacro{\ArcLabelY}{\ArcRadius*sin(\Angle/2)};

  % Center of circumference, line itself, X axis and Y axis
  \coordinate (o) at (0,0);
  \coordinate (p) at (\LineEndX,\LineEndY);
  \coordinate (x) at ($1.1*(\Radius,0)$);
  \coordinate (y) at ($1.1*(0,\Radius)$);

  % Draw the arc itself
  \filldraw[violet, fill opacity=0.3] (o) -- (\ArcRadius,0) arc (0:\Angle:\ArcRadius) -- cycle;
  \node at (\ArcLabelX, \ArcLabelY) [yshift=-0.05cm, xshift=-0.2cm] {$\theta$};

  % Draw the dotted circumference
  \draw[dotted] (o) circle (\Radius);

  % Draw the two axis
  \draw[->, gray] (o) -- (x) node[right]{\textsl{x}};
  \draw[->, gray] (o) -- (y) node[above]{\textsl{y}};
  \draw[-, gray] (o) -- ($-1*(x)$);
  \draw[-, gray] (o) -- ($-1*(y)$);

  % Draw the sine and cosine lines
  \draw[thick, red]  (o) -- ($(\LineEndX, 0)$) node[below, pos=0.5]{$\cos(\theta)$};
  \draw[thick, blue] (p) -- ($(\LineEndX, 0)$) node[right, pos=0.5]{$\sin(\theta)$};

  % Draw the line from the origin to the point, and the point itself
  \draw[thick] (o) -- (p) node[yshift=0.3cm, xshift=-0.3cm, pos=0.6]{$r=1$};
  \filldraw (p) circle (1pt);

  % And finally, the dot at the center
  \filldraw[gray] (o) circle (1pt);
\end{tikzpicturecenter}

Since the radius of the circle (i.e. the hypotenuse of the formed right
triangle) is one, the previous formula remains consistent:
\begin{equation*}
  \sin \left( \theta \right)
  = \frac{\text{opposite}}{\text{hypotenuse}}
  = \frac{\text{opposite}}{1}
  = \text{opposite}
\end{equation*}

% ------------------------------------------------------------------------------
\subsection{Dot product}
% ------------------------------------------------------------------------------

The dot product or scalar product takes two vectors and returns a scalar that
represents the projection of one vector onto the other. In simpler terms, it's a
way of quantifying how aligned is vector $a$ with vector $b$.

The basic formula is the following:
\begin{equation*}
  a \cdot b = a_x b_x + a_y b_y
\end{equation*}

The dot product has a direct relationship with the angle formed by the two
vectors. The dot product of two \textbf{unit vectors} is the cosine of the
angle.
\begin{equation*}
  \hat{a} \cdot \hat{b} = \cos \theta
\end{equation*}

Therefore, if both vectors are \textbf{normalized} (i.e. they are unit vectors),
the returned value will always be in the $[-1,1]$ range.

To calculate the dot product of non-normalized vectors, this formula is used:
\begin{equation*}
  a \cdot b = \|a\| \|b\| \cos \theta
\end{equation*}

The dot product can be expressed as the shadow that $a$ projects over $b$.

\begin{tikzpicturecenter}
  \pgfmathsetmacro{\Radius}{2}
  \pgfmathsetmacro{\ArcRadius}{1}

  \pgfmathsetmacro{\ArcAngleBeta}{40}

  \pgfmathsetmacro{\BetaX}{\ArcRadius*cos(\ArcAngleBeta)}
  \pgfmathsetmacro{\BetaY}{\ArcRadius*sin(\ArcAngleBeta)}

  \pgfmathsetmacro{\LabelX}{\ArcRadius*cos(\ArcAngleBeta/2)}
  \pgfmathsetmacro{\LabelY}{\ArcRadius*sin(\ArcAngleBeta/2)}

  \coordinate (o) at (0,0);
  \coordinate (ub) at (\BetaX, \BetaY);
  \coordinate (a1) at (\Radius,0);
  \coordinate (b1) at ($\Radius*(ub)$);
  \coordinate (a2) at ($1.7*(a1)$);
  \coordinate (b2) at ($1.7*(b1)$);
  \coordinate (b1shadow) at ($(\Radius*\BetaX, 0)$);
  \coordinate (b2shadow) at ($(1.7*\Radius*\BetaX, 0)$);

  % The purple arc
  \filldraw[violet, fill opacity=0.2] (o) -- (\ArcRadius,0) arc (0:\ArcAngleBeta:\ArcRadius) -- cycle;
  \node at (\LabelX, \LabelY) [left] {$\theta$};

  % Dotted circumference
  \draw[dotted] (o) circle (\Radius);

  % Sine dashed lines ("shadows")
  \draw[dashed, gray] (b1) -- (b1shadow);
  \draw[dashed, gray] (b2) -- (b2shadow);

  % Gray A and B lines
  \draw[->, thick, gray] (o) -- (a2) node[right]{$\vec{a}$};
  \draw[->, thick, gray] (o) -- (b2) node[yshift=0.2cm, xshift=0.2cm]{$\vec{b}$};

  % Black UA and UB lines and dots
  \draw[] (o) -- (a1);
  \filldraw (a1) circle (1pt) node[above right]{$\hat{a}$};
  \draw[] (o) -- (b1) node[yshift=0.3cm, xshift=-0.3cm, pos=0.5]{$r = 1$};
  \filldraw (b1) circle (1pt) node[yshift=0.4cm]{$\hat{b}$};

  % Brace to indicate the cosine of the normalized vectors
  \draw[decorate, decoration={brace,amplitude=5pt,mirror}] (o) -- (b1shadow)
  node [midway, yshift=-0.35cm] {$\hat{a} \cdot \hat{b} = \cos(\theta)$};

  % Brace to indicate the cosine of the non-normalized vectors
  \draw[decorate, decoration={brace,amplitude=5pt,mirror,raise=0.7cm}] (o) -- (b2shadow)
  node [midway, yshift=-1.05cm] {$\vec{a} \cdot \vec{b}$};

  % Center of the circumference
  \filldraw (o) circle (1pt);
\end{tikzpicturecenter}

For a more detailed and interactive explanation of the dot product, see Math
Insight \cite{dot_product}.

With this in mind, the dot product can be used to calculate the angle itself.
\begin{align*}
  \cos \theta &= \frac{a \cdot b}{\|a\| \|b\|} \\
  \theta &= \cos^{-1} \left( \frac{a \cdot b}{\|a\| \|b\|} \right)
\end{align*}

A lot of information can be obtained from the dot product. If the dot product is
positive, $a$ has a component in the same direction as $b$. If the dot product
is zero, $a$ and $b$ are perpendicular. If it's negative, $a$ has a component in
the opposite direction of $b$.

\begin{tikzpicturecenter}
  \draw[thin, gray, dotted] (0,0) grid (13,4);

  \draw[<->] (2,3) -- (1,1) -- (3,1.5);
  \node at (2,3) [above] {$a$};
  \node at (3,1.5) [above] {$b$};
  \node at (2,0.5) [draw, rectangle] {$a \cdot b > 0$};

  \draw[<->] (4.5,2.5) -- (6,1) -- (7.5,2.5);
  \node at (4.5,2.5) [above] {$a$};
  \node at (7.5,2.5) [above] {$b$};
  \node at (6,0.5) [draw, rectangle] {$a \cdot b = 0$};

  \draw[<->] (12,1) -- (10,1) -- (9,3);
  \node at (9,3) [above] {$a$};
  \node at (12,1) [above] {$b$};
  \node at (10,0.5) [draw, rectangle] {$a \cdot b < 0$};
\end{tikzpicturecenter}

% ------------------------------------------------------------------------------
% TODO: Fibonacci and Lucas numbers
% ------------------------------------------------------------------------------

% ------------------------------------------------------------------------------
\subsection{Golden ratio}
% ------------------------------------------------------------------------------

The golden ratio is an irrational number with a value of:
\begin{equation*}
  \varphi = \frac{1+\sqrt{5}}{2} = 1.618033988749\dots
\end{equation*}

Two numbers $a$ and $b$ are in the golden ratio (noted $\varphi$) if their ratio
($\frac{a}{b}$) is the same as the ratio of their sum to the larger
number. Assuming $a > b > 0$:
\begin{equation*}
  \frac{a}{b} = \frac{a+b}{a} = \varphi
\end{equation*}

A \textbf{golden rectangle} is a rectangle whose adjacent sides are in the
golden ratio, and it can be used to illustrate the previous formula.

\begin{tikzpicturecenter}
  % Approximation of the golden ratio
  \pgfmathsetmacro{\Phi}{1.618033988};

  % Length of each side of square A
  \pgfmathsetmacro{\SideA}{2};

  % Length of the final rectangle (A and B)
  \pgfmathsetmacro{\SideSum}{\SideA*\Phi};

  % Length of the smaller side of rectangle B
  \pgfmathsetmacro{\SideB}{\SideSum-\SideA};

  % Draw the blue and red rectangles
  \fill[blue, fill opacity=0.2] (0,0) rectangle ++(\SideA,\SideA);
  \fill[red, fill opacity=0.2] (\SideA,0) rectangle ++(\SideB, \SideA);

  % Rectangle lines
  \draw (\SideA,\SideA) -- (\SideA,0);
  \draw[my-green, thick] (0,0) -- (0,\SideA) --
  (\SideSum,\SideA) -- (\SideSum,0) -- cycle;

  % Labels
  \node at ($(0, \SideA*0.5)$) [left, blue] {$a$};
  \node at ($(\SideA*0.5, \SideA)$) [above, blue] {$a$};
  \node at ($(\SideA+\SideB*0.5, \SideA)$) [above, red] {$b$};

  % Brace
  \draw[my-green, decorate, decoration={brace,amplitude=5pt,mirror,raise=0.1cm}]
  (0,0) -- (\SideSum,0) node [midway, yshift=-0.45cm] {$a+b$};
\end{tikzpicturecenter}

The red rectangle with short side $b$ and long side $a$ is itself a golden
rectangle. When placed adjacent to the blue square (with sides of length $a$),
the green rectangle is produced, with long side $a+b$ and short side $a$. This
green rectangle is similar to the red rectangle, and therefore also a golden
rectangle.

This process of adding an adjacent square to the rectangle, and producing a
similar rectangle reminds of the Fibonacci or Lucas sequences. If a Fibonacci
and Lucas number is divided by its immediate predecessor in the sequence, the
quotient approximates to $\varphi$.
\begin{align*}
  \frac{F_{16}}{F_{15}} &= \frac{987}{610} = 1.6180327\dots \\
  \frac{L_{16}}{L_{15}} &= \frac{2207}{1364} = 1.6180351\dots
\end{align*}

% ------------------------------------------------------------------------------
\section{Physics}
% ------------------------------------------------------------------------------

% ------------------------------------------------------------------------------
\subsection{Gravitational force}
% ------------------------------------------------------------------------------
% TODO: Update with improvements in 8dcc/orbit

The gravitational force of each body is calculated with the following formula.
\begin{equation*}
  F = G \frac{m_1m_2}{r^2}
\end{equation*}

Where $G$ is the gravitational constant, $m_1$ and $m_2$ are the mass of
each body, and $r$ is the distance between the objects.

The effect of a force is to accelerate the body. The relationship is the
following.
\begin{equation*}
  F = m a
\end{equation*}

Where $F$ is the force, $m$ is the mass and $a$ is the acceleration of
the body. Therefore, the acceleration can be calculated from the force with the
following formula.
\begin{equation*}
  a = \frac{F}{m}
\end{equation*}

The force has a direction. It acts towards the direction of the line joining
the centres of the two bodies. We can get the X and Y coordinates of the
acceleration with some trigonometry.
\begin{align*}
  a_x &= a \cos \theta \\
  a_y &= a \sin \theta
\end{align*}

Where $a_x$ and $a_y$ are the X and Y accelerations, $a$ is the
acceleration, and $\theta$ is the angle that the line joining the centers make
with the horizontal.

% ------------------------------------------------------------------------------
\section{Formal logic}
% ------------------------------------------------------------------------------

Formal logic uses a formal (i.e. abstract) approach to study reasoning. It
replaces concrete expressions with abstract symbols to examine the logical form
of arguments independent of their concrete content.

For a good resource on formal logic, see \textit{Applied Discrete Structures}
\cite{ads}.

% ------------------------------------------------------------------------------
\subsection{Small glossary}
% ------------------------------------------------------------------------------

These are some of the terms that will be used throughout this section.

\begin{description}
  \item[Proposition] Sentence to which one and only one of the terms
    \textit{true} or \textit{false} can be meaningfully applied. The most
    commonly used letters to represent propositions are $p$, $q$ and $r$.
  \item[Condition] First proposition of a conditional.
  \item[Conclusion] Second proposition of a conditional.
  \item[Tautology] An expression involving logical variables that is true in all
    cases. The number 1 is used to symbolize a tautology.
  \item[Contradiction] An expression involving logical variables that is false
    in all cases. The number 0 is used to symbolize a contradiction.
  \item[Equivalence] Two propositions $p$ and $q$ are equivalent if and only if
    $p \leftrightarrow q$ is a tautology. See Table \ref{tbl:logic_op} and Table
    \ref{tbl:logic_osym}.
  \item[Implication] A proposition $p$ implies $q$ if $p \to q$ is a
    tautology. See Table \ref{tbl:logic_op} and Table \ref{tbl:logic_osym}.
\end{description}

% ------------------------------------------------------------------------------
\subsection{Operation notation and priority}
% ------------------------------------------------------------------------------

The following table shows the notation used for some logical operations. It is
ordered from higher to lower operator precedence.

\begin{table}[h]
  \caption{\label{tbl:logic_op}Logical operators}
  \begin{center}
    \begin{tabular}{| c | c | c |}
      \hline
      Notation & Operation name & English form \\
      \hline
      $\neg p$              & Logical negation        & Not $p$ \\
      $p \land q$           & Logical conjunction     & $p$ and $q$ \\
      $p \lor q$            & Logical disjunction     & $p$ or $q$ \\
      $p \to q$             & Conditional operation   & If $p$, then $q$ \\
      $p \leftrightarrow q$ & Biconditional operation & $p$ if and only if $q$ \\
      \hline
    \end{tabular}
  \end{center}
\end{table}

The \textit{Venn diagrams} might help understand some of these operators.

\medskip

The following table shows the symbols for other terms in this section.

\begin{table}[h]
  \caption{\label{tbl:logic_osym}Other symbols}
  \begin{center}
    \begin{tabular}{| c | c | c |}
      \hline
      Notation & Operation name & English form \\
      \hline
      $p \iff q$     & Equivalence & $p$ is equivalent to $q$ \\
      $p \implies q$ & Implication & $p$ implies $q$ \\
      \hline
    \end{tabular}
  \end{center}
\end{table}

% ------------------------------------------------------------------------------
\subsection{Conditional vs. biconditional}
% ------------------------------------------------------------------------------

A simple conditional can be expressed with the structure ``If
\textit{Condition}, then \textit{Conclusion}''. A conditional statement is meant
to be interpreted as a guarantee; if the condition is true, then the conclusion
is expected to be true. It says no more and no less.

The biconditional, however, indicates that the two propositions depend on each
other. If one is true, the other must be true.

This slightly modified example from \textit{Applied Discrete Structures}
(pp. 41-42) might help illustrate the difference between the conditional and
biconditional operations.

Assume your instructor told you ``If you receive a grade of 95 or better in the
final examination, then you will receive an A in this course''. This is a simple
condition, and this is its truth table.

\medskip
\begin{center}
  \begin{tabular}{| c | c | c |}
    \hline
    $\text{Grade} \geq 95$ & $\text{Course} = \text{A}$ & $p \to q$ \\
    \hline
    0 & 0 & 1 \\
    0 & 1 & 1 \\
    1 & 0 & 0 \\
    1 & 1 & 1 \\
    \hline
  \end{tabular}
\end{center}
\medskip

Where, in order to keep the table simple, $p$ and $q$ represent the two
propositions respectively.

Note that the last column essentially indicates if the instructor told the
truth. Since the condition was simple, and not biconditional, the only case in
which he would have lied is if we had a score greater or equal to 95, but we
didn't get the A. If we didn't reach the score of 95, we \textit{can} still get
an A on this course, since he didn't make a promise related to us getting a
grade below 95.

However, assume the instructor told you ``You will receive an A in this course
\textit{if and only if} you receive a grade of 95 or better in the final
examination''. In this case, it's a biconditional promise, since he is also
implying that if you don't receive a grade of at least 95 in the final
examination, you will not be able to get the A in this course. This is the truth
table for this new promise.

\medskip
\begin{center}
  \begin{tabular}{| c | c | c |}
    \hline
    $\text{Grade} \geq 95$ & $\text{Course} = \text{A}$ & $p \leftrightarrow q$ \\
    \hline
    0 & 0 & 1 \\
    0 & 1 & 0 \\
    1 & 0 & 0 \\
    1 & 1 & 1 \\
    \hline
  \end{tabular}
\end{center}
\medskip

Again, the last column just indicates if the instructor told the truth. If your
grade was below 95, and you didn't receive the A in this course, the instructor
would have told the truth. However, if one of these propositions did not match
the other, the instructor would have lied.

% ------------------------------------------------------------------------------
\subsection{Tautologies, contradictions and equivalences}
% ------------------------------------------------------------------------------

The following list presents some tautologies (i.e. propositions that are always
true). These propositions should usually be avoided because they provide no
information.

\begin{itemize}
  \item $p \lor \neg p$
  \item $p \land q \to p$
  \item $p \to p \lor q$
\end{itemize}

The following list presents some contradictions (i.e. propositions that are
always false). Just like with tautologies, these propositions should usually be
avoided.

\begin{itemize}
  \item $p \land \neg p$
\end{itemize}

The following list presents some groups of propositions that share the same
meaning. These become tautologies when using a biconditional operator, but I
decided to separate them from the previous list because they serve as a list of
propositions that could be expressed with an alternate form.

\begin{itemize}
  \item $\neg (p \land q) \ \iff \ \neg p \lor \neg q$
  \item $(p \land q) \lor (\neg p \land q) \ \iff \ q$
\end{itemize}

% ------------------------------------------------------------------------------
\section{Modulus operation}
% ------------------------------------------------------------------------------

The modulus of two numbers is the remainder of it's integer division. The
modulus of two numbers could be defined as follows.
\begin{equation*}
  \lfloor a / b \rfloor \times b + a \bmod b = a
\end{equation*}

Where $\lfloor a / b \rfloor$ indicates the integer division of $a$ and $b$.

% ------------------------------------------------------------------------------
\subsection{Equivalences}
% ------------------------------------------------------------------------------

These equivalences might be useful when dealing with modulus operators that only
support positive values, for example.

\bigskip

Given the following function, that returns the modulus of two positive values,
\begin{equation*}
  \text{AbsMod}(a, b) = |a| \bmod |b|
\end{equation*}
the following conditional formula can be used for determining the modulus of any
positive and negative combination.
\begin{equation*}
  a \bmod b =
  \begin{cases}
    \text{AbsMod}(a, b),     & a \geq 0 \land b \geq 0 \\
    b + \text{AbsMod}(a, b), & a \geq 0 \land b < 0 \\
    b - \text{AbsMod}(a, b), & a < 0 \land b \geq 0 \\
    -\text{AbsMod}(a, b),    & a < 0 \land b < 0
  \end{cases}
\end{equation*}

\bigskip

The modulus of $a$ and $b$ is equal to the negation of the modulus of $-a$ and
$-b$.
\begin{equation*}
  a \bmod  b \iff - \left( -a \bmod -b \right)
\end{equation*}

This can be used for converting the divisor and dividend to negative, if needed.
\begin{align*}
  a  \bmod -b & \iff - \left( -a \bmod  b \right) \\
  -a \bmod  b & \iff - \left(  a \bmod -b \right) \\
  -a \bmod -b & \iff - \left(  a \bmod  b \right)
\end{align*}

\bigskip

The modulus of $a$ and $b$ is equal to the divisor ($b$) minus the modulus of
the negated dividend and the unchanged divisor.
\begin{equation*}
  a \bmod b \iff b - (-a \bmod b)
\end{equation*}

This can be used for converting the dividend to positive, if needed.
\begin{equation*}
  -a \bmod b \iff b - a \bmod b
\end{equation*}

% ------------------------------------------------------------------------------
\section{Color conversion}
% ------------------------------------------------------------------------------

% ------------------------------------------------------------------------------
\subsection{Value ranges}
% ------------------------------------------------------------------------------

An RGB color has values in the $[0..255]$ range, while in an HSV color the
\textit{hue} is in the $[0..360]$ range and the \textit{saturation} and
\textit{value} are in the $[0..1]$ range, although they might be represented as
percentages.

% ------------------------------------------------------------------------------
\subsection{RGB to HSV}
% ------------------------------------------------------------------------------

First, the RGB values need to be normalized to the $[0..1]$ range.
\begin{align*}
  R' &= \frac{R}{255} \\
  G' &= \frac{G}{255} \\
  B' &= \frac{B}{255}
\end{align*}

Then, the maximum and minimum RGB values are calculated, along with its
difference.
\begin{align*}
  C_{max} &= \max(R', G', B') \\
  C_{min} &= \min(R', G', B') \\
  \Delta  &= C_{max} - C_{min}
\end{align*}

To calculate the \textit{hue}, the following conditional formula is used.
\begin{equation*}
  H =
  \begin{cases}
    0^\circ,                                                    & \Delta = 0 \\
    60^\circ \times \left(\frac{G'-B'}{\Delta} \bmod 6 \right), & C_{max} = R' \\
    60^\circ \times \left(\frac{B'-R'}{\Delta} + 2 \right),     & C_{max} = G' \\
    60^\circ \times \left(\frac{R'-G'}{\Delta} + 4 \right),     & C_{max} = B'
  \end{cases}
\end{equation*}

To calculate the \textit{saturation}, the following formula is used.
\begin{equation*}
  S =
  \begin{cases}
    0,                      & C_{max} = 0 \\
    \frac{\Delta}{C_{max}}, & C_{max} \neq 0 \\
  \end{cases}
\end{equation*}

Finally, since $C_{max}$ is already normalized, it can be used directly as the
\textit{value} component.
\begin{equation*}
  V = C_{max}
\end{equation*}

% ------------------------------------------------------------------------------
\subsection{HSV to RGB}
% ------------------------------------------------------------------------------

Calculate the \textit{chroma} by multiplying the \textit{saturation} and the
\textit{value}.
\begin{equation*}
  C = S \times V
\end{equation*}

Then, the $X$ value is calculated, which will be used as a component in the
initial RGB color below.
\begin{gather*}
  H' = \frac{H}{60^\circ} \\
  X = C \times \left( 1 - \left| H' \bmod 2 - 1 \right| \right)
\end{gather*}

Note that $H'$ must be an integer for the modulus operation.

The \textit{chroma} and $X$ values will be used for the initial RGB values
depending on the \textit{hue} with this conditional formula.
\begin{equation*}
  \left( R', G', B' \right) =
  \begin{cases}
    (C, X, 0), & 0^\circ \leq H < 60^\circ \\
    (X, C, 0), & 60^\circ \leq H < 120^\circ \\
    (0, C, X), & 120^\circ \leq H < 180^\circ \\
    (0, X, C), & 180^\circ \leq H < 240^\circ \\
    (X, 0, C), & 240^\circ \leq H < 300^\circ \\
    (C, 0, X), & 300^\circ \leq H < 360^\circ
  \end{cases}
\end{equation*}

The value of $H'$ can be used in the conditions instead of the \textit{hue}, but
I consider this form more visual.

To find the real RGB values, $m$ has to be added to each component to match the
HSV \textit{value}.
\begin{gather*}
  m = V - C \\
  (R, G, B) = (R' + m, G' + m, B' + m)
\end{gather*}

% ------------------------------------------------------------------------------
% Bibliography
% ------------------------------------------------------------------------------

\clearpage
\phantomsection
\addcontentsline{toc}{section}{References}

\begin{thebibliography}{9}
\bibitem{vector_introduction}
  Frank D and Nykamp DQ. \emph{An introduction to vectors}. Math
  Insight. Retrieved 23 May 2024, from
  \url{http://mathinsight.org/vector_introduction}
\bibitem{magnitude_vector}
  Nykamp DQ. \emph{Magnitude of a vector definition}. Math
  Insight. Retrieved 17 Jun 2024, from
  \url{https://mathinsight.org/definition/magnitude_vector}
\bibitem{unit_vector}
  Wikipedia. \emph{Unit vector}. Retrieved 23 May 2024, from
  \url{https://en.wikipedia.org/wiki/Unit_vector}
\bibitem{dot_product}
  Nykamp DQ. \emph{The dot product}. Math Insight. Retrieved 23 May 2024,
  from \url{https://mathinsight.org/dot_product}
\bibitem{ads}
  Al Doerr and Ken Levasseur. \emph{Applied Discrete Structures}.
  3rd ed. pp. 39-72.
\end{thebibliography}

\end{document}
