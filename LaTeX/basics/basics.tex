\documentclass{article}

\title{Scratch - Basics of \LaTeX}
\author{8dcc}
\date{2024}

% ------------------------------------------------------------------------------
% Packages
% ------------------------------------------------------------------------------

% Clickable table of contents
\usepackage[hidelinks]{hyperref}
\hypersetup{
  linktoc=all,     % Link sections and subsections
}

% Prevent indentation after lists
\usepackage{noindentafter}
\NoIndentAfterEnv{itemize}
\NoIndentAfterEnv{enumerate}

% Code highlighting.
\usepackage{listings}
\lstset{
  % Showing spaces
  showspaces=false,
  showstringspaces=false,
  showtabs=false,
  % Indentation and breaks
  tabsize=4,
  breaklines=true,
  breakatwhitespace=true,
  columns=flexible,
  % Show left, right, top and bottom borders
  frame=tblr,
  % Misc
  aboveskip=3mm,
  belowskip=3mm,
  basicstyle={\small\ttfamily},
}

% Different monospace font for code blocks (listings)
\usepackage{inconsolata}

% ------------------------------------------------------------------------------
% Document start
% ------------------------------------------------------------------------------

\begin{document}

\maketitle
\newpage

\tableofcontents
\newpage

% ------------------------------------------------------------------------------
\section{Introduction}
% ------------------------------------------------------------------------------

Simple scratch file for testing the basics of \LaTeX.

% ------------------------------------------------------------------------------
\section{Text formats}
% ------------------------------------------------------------------------------

Here is a list of text formats:

\begin{itemize}
  \item \textbf{Bold}
  \item \textit{Italic}
  \item \underline{Underlined}
  \item \emph{Emphasized}
  \item \textsc{Small caps}
  \item \textrm{Roman}
  \item \textsl{Slanted}
  \item \texttt{Typewriter}
\end{itemize}

You can \underline{\textbf{\textit{combine}}} these effects too.

To quote a \LaTeX command, you can use the \verb|verb| command. You can also use
\verb|\begin{verbatim}| and \verb|\end{verbatim}|.

This is a \href{http://github.com/8dcc}{\textbf{Link}}, thanks to the
``hyperref'' package.

% ------------------------------------------------------------------------------
\section{Lists}
% ------------------------------------------------------------------------------

This is an unordered list:

\begin{itemize}
  \item I am the first item.
  \item I am the second item.
\end{itemize}

% TODO: Why is this indented?
This is an ordered list:

\begin{enumerate}
  \item I am the first item.
  \item I am the second item.
\end{enumerate}

This is some C code:

\begin{lstlisting}[language=C]
  while (a != b) {
    // ...
  }
\end{lstlisting}

% ------------------------------------------------------------------------------

\newpage
\begin{thebibliography}{9}
\bibitem{wikipedia_example}
  Wikipedia. \textit{Example Title}. Retrieved 1 Jan 2024, from
  \url{https://en.wikipedia.org/wiki/Example}
\bibitem{mathinsight_dotproduct}
  Nykamp DQ. \textit{The dot product}. From Math Insight. Retrieved 23 May 2024, from
  \url{https://mathinsight.org/dot_product}
\bibitem{reversing}
  Eldad Eilam. (2005). \textit{Reversing: Secrets of Reverse Engineering} (pp. 56-57).
\end{thebibliography}

\end{document}
