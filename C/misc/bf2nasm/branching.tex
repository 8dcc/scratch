\documentclass{amsart}

\title{bf2nasm - Smarter branching}
\author{8dcc}

% Graphs
\usepackage{tikz}

% Code highlighting
\usepackage{listings}

% Hyperlinks
\usepackage[hidelinks]{hyperref}

% Code highlighting.
\usepackage{listings}
\lstset{
  % Showing spaces
  showspaces=false,
  showstringspaces=false,
  showtabs=false,
  % Indentation and breaks
  tabsize=4,
  breaklines=true,
  breakatwhitespace=true,
  columns=flexible,
  % Show left, right, top and bottom borders
  frame=tblr,
  % Misc
  aboveskip=3mm,
  belowskip=3mm,
  basicstyle={\small\ttfamily},
}

% Remove author and extra info from the headers
\pagestyle{plain}

% New environment adding spacing for TikZ pictures
\newenvironment{tikzpicturecenter}
{\begin{center}\begin{tikzpicture}}
    {\end{tikzpicture}\end{center}}

\begin{document}
\maketitle

\section{Pretested and posttested loops}

When I was reading \textit{Reversing: Secrets of Reverse Engineering}
\cite{reversing}, I found this part specially interesting:

\begin{quote}
  The most common high-level loop construct is the pretested loop, where the
  loop's condition is tested before the loop's body is executed. The problem
  with this construct is that it requires an extra unconditional jump at the end
  of the loop's body in order to jump back to the beginning of the loop (for
  comparison, posttested loops only have a single conditional branch instruction
  at the end of the loop, which makes them more efficient). Because of this, it
  is common for optimizers to convert pretested loops to posttested loops. In
  some cases, this requires the insertion of an if statement before the
  beginning of the loop, so as to make sure the loop is not entered when its
  condition isn't satisfied.
\end{quote}

The compiler will sometimes optimize the following pretested loop:

\begin{lstlisting}[language=C]
  while (a != b) {
    ...
  }
\end{lstlisting}

Into this posttested loop inside a conditional.

\begin{lstlisting}[language=C]
  if (a != b) {
    do {
      ...
    } while (a != b);
  }
\end{lstlisting}

The assembly for the first loop would be something like:

\begin{lstlisting}
  loop:
    cmp A, B
    je done
    ...
    jmp loop
  done:
\end{lstlisting}

And the assembly for the second loop could be translated literally to something
like:

\begin{lstlisting}
    cmp A, B
    je done
  loop:
    ...
    cmp A, B
    jne loop
  done:
\end{lstlisting}

As mentioned in the quote, on the first loop there is a conditional jump and an
unconditional one, while in the second one, the first condition is tested once
before the loop, and the loop only has one conditional jump. If the condition
fails on the second loop, the jump will not be performed and the execution will
continue at the \texttt{done} label.

\section{Brainfuck loops}

In brainfuck, loops are defined with square brackets. The code inside the loop
will be ran as long as the value at the current cell is not zero. At first, I
thought this was a posttested loop, but this is wrong. The loops are pretested,
meaning that the loop will be jumped over entirely if the value at the current
cell is not zero.

When the program encounters a loop start, the following assembly is generated:

\begin{lstlisting}
    jmp .check_N
  .loop_N:
    ...
  .check_N:
    cmp byte [rcx], 0
    jnz .loop_N
\end{lstlisting}

Where $N$ is the loop counter.

That way, we even avoid the first comparison from the other example, and we jump
directly to the ``end'' of the loop, where the condition is checked. Now we only
make an ``extra'' unconditional jump once, for checking the condition the first
time when entering the loop.

% -------------------------------------------------------------------------------

\bigskip

\begin{thebibliography}{9}
\bibitem{reversing}
  Eldad Eilam. (2005). \textit{Reversing: Secrets of Reverse Engineering} (pp. 56-57).
\end{thebibliography}

\end{document}
